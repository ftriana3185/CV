\documentclass[spanish]{resume} % Use the custom resume.cls style
\usepackage[left=0.75in,top=0.6in,right=0.75in,bottom=0.6in]{geometry} % Document margins
\usepackage[spanish]{babel}
\usepackage{hyperref}

\name{Felipe Triana Casta\~neda} % Your name
\address{Calle 6B No-9-26 \\ Madrid, Cundinamarca, Colombia 250030} % Your address
\address{\url{https://ftrianakast-cedar-14.herokuapp.com/}}
\address{(+57)~$\cdot$~310~$\cdot$~8187426 \\ ftrianakast@gmail.com} % Your phone number and email


\begin{document}


%----------------------------------------------------------------------------------------
% SUMMARY SECTION
%----------------------------------------------------------------------------------------

\begin{rSection}{Resumen}

Ingeniero de software con buenas habilidades comunicativas y de trabajo en grupo. Entusiasta en temas de arquitectura de software y ciencia de datos. Como ingeniero de software capaz de definir y llevar a cabo arquitecturas basadas en m\'ultiples frameworks y lenguajes de programaci\'on, y como entusiasta de ciencia de datos estoy familiarizado con t\'ecnicas estad\'isticas, de an\'alisis de redes y de machine learning.

\end{rSection}

%----------------------------------------------------------------------------------------
% EDUCATION SECTION
%----------------------------------------------------------------------------------------

\begin{rSection}{Educaci\'on}

{\bf Universidad de los Andes, Bogot\'a D.C} \hfill {\em Agosto 2013} \\
Ingenier\'ia de Sistemas \& Computaci\'on \\
Tesis SkinHealth, un sistema para la detecci\'on de enfermedades de la piel utilizando la Ontolog\'ia Ontoderm.

{\bf Colegio Seminario San Juan Ap\'ostol, Facatativ\'a} \hfill {\em 2008} \\
Bachiller Acad\'emico.

\end{rSection}

%----------------------------------------------------------------------------------------
% WORK EXPERIENCE SECTION
%----------------------------------------------------------------------------------------
\begin{rSection}{Experiencia}

\begin{rSubsection}{Globant Latin America}{May 2015 - October 2015}{Software Engineer}{Bogot\'a D.C}
\item Python dev. Desarrollar scripts para migración de grandes datos utilizando Python.
\item AEM developer: componentes, workflows y servicios OSGi.
\item Diseñador de webservices y desarrollador sobre la plataforma Apache Sling.  
\item Ingeniero Backend usando lenguaje Java.
\item Experiencia con un equipo internacional de desarrolladores.
\item Trabajé en dos proyectos como referente técnico: proyecto Hewllett Packard y proyecto USANA.
\end{rSubsection}


\begin{rSubsection}{Prodigious Latin America}{May 2015 - October 2015}{Senior Software Engineer}{Bogot\'a D.C| San Francisco}
\item AEM developer: componentes, workflows y servicios OSGi.
\item Diseñador de webservices y desarrollador sobre la plataforma Apache Sling.  
\item Ingeniero Backend usando lenguaje Java.
\item Experiencia con un equipo internacional de desarrolladores.
\item Trabajé en dos proyectos como referente técnico: proyecto Hewllett Packard y proyecto USANA.
\end{rSubsection}


\begin{rSubsection}{Aentropico SAS}{Mayo 2014 - Octubre 2014}{CTO}{Bogot\'a D.C | R\'io de Janeiro}
\item L\'ider tecnol\'ogico de la organizaci\'on y representante de la misma. Encargado de dar las directrices de tecnolog\'ia que soportan el negocio.
\item Dise\~no de un API Rest para el consumo de la plataforma de an\'alisis de datos.
\item Ingenier\'ia de software utilizando diferentes lenguajes de programaci\'on para llevar acabo la arquitectura mencionada: NodeJS, R y Java.
\item Desarrollador frontend utilizando Javascript (AngularJS Framework), CSS3 (preprocesador Stylus), HTML5 (motor de templating Jade).
\item Dise\~no de base de datos utilizando tecnolog\'ia NoSQL (MongoDB).
\item Ingeniero de infraestructura encargado del despliegue (AWS, Docker, Heroku), de la seguridad (OAuth 2.0) y de la escalabilidad de la plataforma.
\item Ingeniero de requerimientos. A partir de requerimientos de negocio discernir requerimientos funcionales.
\item Scrum master. Planeaci\'on de sprints y responsable de fechas de entrega.
\end{rSubsection}

%------------------------------------------------

\begin{rSubsection}{Aentropico SAS}{Agosto 2013 - Mayo 2014}{Full Stack Developer}{Bogot\'a D.C | R\'io de Janeiro}
\item Dise\~no arquitectural de una plataforma que provee herramientas de An\'alisis Predictivo (machine learning) y de algor\'itmica avanzada para cualquier tipo de negocios.
\item Dise\~no de un API Rest para el consumo de la plataforma de an\'alisis.
\item Ingenier\'ia de software utilizando diferentes lenguajes de programaci\'on para llevar acabo la arquitectura mencionada: NodeJS, R y Java.
\item Desarrollador frontend utilizando Javascript (AngularJS Framework), CSS3 (preprocesador Stylus), HTML5 (motor de templating Jade).
\item Dise\~no de base de datos utilizando tecnolog\'ia NoSQL (MongoDB).
\item Ingeniero de infraestructura encargado del despliegue (AWS, Docker, Heroku), de la seguridad (OAuth 2.0) y de la escalabilidad de la plataforma.
\item Ingeniero de requerimientos. A partir de requerimientos de negocio discernir requerimientos funcionales.
\item Scrum master. Planeaci\'on de sprints y responsable de fechas de entrega.
\end{rSubsection}

%------------------------------------------------

\begin{rSubsection}{IBM}{Agosto 2012 - Diciembre 2012}{Analista T\'ecnico}{Bogot\'a D.C}
\item Ingeniero de integraci\'on de aplicaciones encargado del levantamiento de requerimientos de integraci\'on e implementaci\'on de los mismos para una empresa del sector bancario. El trabajo estuvo enmarcado dentro de una metodolog\'ia tradicional de desarrollo certificada nivel 5 dentro del modelo de madurez CMMI.
\item Implementaci\'on de servicios de integraci\'on utilizando la plataforma Websphere Message Broker.
\item Implementaci\'on de adaptadores Java para aplicaciones legado.
\end{rSubsection}

%------------------------------------------------

\begin{rSubsection}{SoftOne SAS}{Junio 2012 - Agosto 2012}{Desarrollador Freelancer}{Bogot\'a D.C}
\item Desarrollo de servicios web sobre el framework JEE para el manejo de los canales IVR y Kioskos de una empresa financiera.
\item Desarrollo de una aplicaci\'on m\'ovil sobre la plataforma J2ME para la distribuci\'on de pedidos.
\item Los dos proyectos estuvieron enmarcados dentro de una metodolog\'ia \'agil de desarrollo (Scrum).
\end{rSubsection}

\end{rSection}

%----------------------------------------------------------------------------------------
% TECHNICAL STRENGTHS SECTIONx
%----------------------------------------------------------------------------------------

\begin{rSection}{Habilidades t\'ecnicas}

\begin{tabular}{ @{} >{\bfseries}l @{\hspace{6ex}} l }
Arquitectura de Software & SOA (servicios Rest y SOAP), DDD, MDD, patrones de dise\~no \\
Lenguajes de programaci\'on & Java, Scala, Pyhton, Javascript, Haskell \\
Librer\'as & Scalaz \\ 
Frameworks y plataformas & Akka, Spray.io, Play Framework, Spring, JEE, NodeJS, ExpressJS \\
Frontend & AngularJS, BackboneJS, CSS3, HTML5, Jade, Stylus, Grunt, Gulp \\
Protocolos \& APIs & XML, JSON, SOAP, REST \\
Databases & MySQL, PostgreSQL, MongoDB, Redis, Neo4j \\
Tools & Maven, SBT, Npm, Git, SVN
\end{tabular}

\end{rSection}

%----------------------------------------------------------------------------------------
% PROJECT MANAGEMENT SKILLS
%----------------------------------------------------------------------------------------
\begin{rSection}{Habilidades de administraci\'on de proyectos}

\begin{tabular}{ @{} >{\bfseries}l @{\hspace{6ex}} l }
Metodolog\'ias & Scrum, XP, TSP. \\
Comunicaci\'on & Buena expresi\'on oral y escrita.
\end{tabular}

\end{rSection}


%----------------------------------------------------------------------------------------
% LANGUAGES
%----------------------------------------------------------------------------------------
\begin{rSection}{Idiomas}

\begin{tabular}{ @{} >{\bfseries}l @{\hspace{6ex}} l }
Ingl\'es & Habla bien, escribe bien, lee bien. \\
Espa\~nol & Nativo.
\end{tabular}

\end{rSection}

%----------------------------------------------------------------------------------------
% RECONOCIMIENTOS
%----------------------------------------------------------------------------------------
\begin{rSection}{Reconocimientos}

{\bf Beca quiero estudiar Uniandes} \hfill {\em 2009} \\
Universidad de los Andes, Bogot\'a D.C.
Otorgada a los ICFES m\'as sobresalientes del pa\'is.

{\bf Mejor Bachiller Acad\'emico COLSEM} \hfill {\em 2008} \\
Colegio Seminario San Juan Ap\'ostol.

\end{rSection}


%----------------------------------------------------------------------------------------
% CURSOS
%----------------------------------------------------------------------------------------
\begin{rSection}{Cursos - Conferencias}


{\bf Coursera, Standford University, United States} \hfill {\em Abril 2016} \\
Machine Learning.
Este curso provee una introducción extensa al área de Machine Learning, datamining y reconocimiento de patrones estadísticos. Entre los temas a tratar se pueden encontrat: (i) Supervised learning (parametric/non-parametric algorithms, support vector machines, kernels, redes neuronales). (ii) Unsupervised learning (clustering, dimensionality reduction, recommender systems, deep learning). (iii) Mejores practicas en machine learning(bias/variance theory; procesos de innovación en machine learning e inteligencia artificial).


This course provides a broad introduction to machine learning, datamining, and statistical pattern recognition. Topics include: (i) Supervised learning (parametric/non-parametric algorithms, support vector machines, kernels, neural networks). (ii) Unsupervised learning (clustering, dimensionality reduction, recommender systems, deep learning). (iii) Best practices in machine learning (bias/variance theory; innovation process in machine learning and AI).


{\bf Coursera, \'Ecole Polytechnique F\'ed\'erale de Lausanne, Switzerland} \hfill {\em Mayo 2015} \\
Principles of Reactive Programming.
Elementos claves para escribir programas reactivos de una manera componible y la manera de aplicar estos elementos en la construcción de message-driven systems que son escalables y resilentes. 


{\bf Coursera, Escuela Polit\'ecnica Federal de Lausana, Suiza} \hfill {\em Septiembre 2014} \\
Functional Programming Principles in Scala.
Curso sobre temas introductorios y avanzados de programaci\'on funcional usando Scala.

{\bf Speaker Javascript Conference, Medell\'in Colombia} \hfill {\em Octubre 2013} \\
Conferencista invitado en la conferencia anual de Javascript representando a Aentr\'opico SAS.
Tema: explorador de visualizaciones, una danza entre nodeJS y D3.js.

{\bf edX, Universidad de Berkeley, California} \hfill {\em Septiembre 2010} \\
Artifitial Intelligence.
Principios de inteligencia artificial.

\end{rSection}


\end{document}