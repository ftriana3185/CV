\documentclass[spanish]{resume} % Use the custom resume.cls style
\usepackage[left=0.75in,top=0.6in,right=0.75in,bottom=0.6in]{geometry} % Document margins
\usepackage[spanish]{babel}
\usepackage{hyperref}

\name{Felipe Triana Casta\~neda} % Your name
\address{Calle 6B No-9-26 \\ Madrid, Cundinamarca, Kolumbien 250030} % Your address
\address{\url{https://ftrianakast-cedar-14.herokuapp.com/}}
\address{(+57)~$\cdot$~310~$\cdot$~8187426 \\ ftrianakast@gmail.com} % Your phone number and email


\begin{document}


%----------------------------------------------------------------------------------------
% SUMMARY SECTION
%----------------------------------------------------------------------------------------

\begin{rSection}{Zusammenfassung}

Software-Ingenieur mit guten kommunikativen F{\"a}higkeiten und Teamfhigkeit. Begeister von Themen der Softwarearchitektur, des Data Science, der funktionalen Programmierung, der verteilten Systeme und der Kategorientheorie. Als Software-Ingenieur bin ich bef{\"a}higt, Architekturen zu definieren und auszuf{\"u}hren, die auf multiplen Frameworks und Programmiersprachen basieren, und als Anh{\"a}nger des Data Science bin ich mit Statistiktechniken, Netzwerkanalyse und maschinellem Lernen vertraut.

\end{rSection}

%----------------------------------------------------------------------------------------
% INTERESTS SECTION
%----------------------------------------------------------------------------------------
\begin{rSection}{Interessen}

\item Verteilte Systeme.
\item Funktionale Programmierung (Scala, Haskell).
\item Kategorientheorie.
\item Data Science.
\item Wahrscheinlichkeitsrechnung und Statistik.

\end{rSection}

%----------------------------------------------------------------------------------------
% EDUCATION SECTION
%----------------------------------------------------------------------------------------

\begin{rSection}{Ausvildung}

{\bf Universit{\"a}t der Anden, Bogot\'a D.C} \hfill {\em August 2013} \\
Systemtechnik \& Informatik \\
Diplomarbeit Hautgesundheit [Anm. des {\"U}bers.: Auf Englisch: SkinHealth], ein System zur Erkennung von Hautkrankheiten unter Einsatz der Ontologie ONTODerm.

{\bf Schule Seminar San Juan Ap\'ostol, Facatativ\'a} \hfill {\em 2008} \\
Allgemeine Hochschulreife.

\end{rSection}

%----------------------------------------------------------------------------------------
% WORK EXPERIENCE SECTION
%----------------------------------------------------------------------------------------
\begin{rSection}{Berufserfahrung}

\begin{rSubsection}{Prodigious Latin America}{M{\"a}rz 2015 - heute}{Leitender Software-Ingenieur}{Bogot\'a D.C| San Francisco}
\item AEM Entwickler: Komponenten, Workflows und OSGi Services.
\item Gestalter von Webservices und Entwickler auf der Grundlage der Plattform Apache Sling.  
\item Backend Ingenieur unter Einsatz der Programmiersprache Java.
\item Erfahrung mit einem internationalen Entwicklerteam.
\item Mitarbeit an drei Projekten als technische Bezugsgröße: Hewlett Packard, Projekt USANA, Projekt Renault-Nissan.
\end{rSubsection}

[Anm. des Übers.: Ende der ersten Seite des Lebenslaufs]

\begin{rSubsection}{Aentropico SAS}{Mai 2014 - Oktober 2014}{Technikvorstand (CTO)}{Bogot\'a D.C | R\'io de Janeiro}
\item Technologieleiter des Unternehmens und Bevollm{\"a}chtigter des selbigen. Beauftragt, die Weisungen f{\"u}r die Technik vorzugeben, die den Gesch{\"a}ftsbetrieb unterst{\"u}tzen.
\item Entwicklung eines API Rest f{\"u}r den Verbrauch der Datenanalyse-Plattform.
\item Softwareentwicklung unter Einsatz verschiedener Programmiersprachen, um die genannte Architektur auszuf{\"u}hren: NodeJS, R und Java.
\item Frontend Entwickler unter Einsatz von Javascript (AngularJS Framework), CSS3 (Vorprozessor Stylus), HTML5 (Motor der Templating Jade).
\item Entwicklung von Datenbanken unter Einsatz von NoSQL-Technologie (MongoDB).
\item Konstrukteur der Infrastruktur, verantwortlich f{\"u}r die Bereitstellung (AWS, Docker, Heroku), Sicherheit (OAuth 2.0) und die Stabilität der Plattform.
\item Anforderungsingenieur. Ausgehend von den Anforderungen des Geschäftsbetriebs funktionelle Anforderungen erkennen.
\item Scrum Master. Planung von Sprints und verantwortlich für die Liefertermine.
\end{rSubsection}

%------------------------------------------------

\begin{rSubsection}{Aentropico SAS}{August 2013 - Mai 2014}{Full-Stack Softwareentwickler}{Bogot\'a D.C | R\'io de Janeiro}
\item Entwicklung der Architektur einer Plattform, die für jede Art der Gesch{\"a}ftst{\"a}tigkeit Instrumente zur Vorhersageanalyse (maschinelles Lernen) und fortgeschrittene Algorithmik bereitstellt.
\item Entwicklung eines API Rest für den Verbrauch der Datenanalyse-Plattform.
\item Softwareentwicklung unter Einsatz verschiedener Programmiersprachen, um die genannte Architektur auszuf{\"u}hren: NodeJS, R und Java.
\item Frontend Entwickler unter Einsatz von Javascript (AngularJS Framework), CSS3 (Vorprozessor Stylus), HTML5 (Motor der Templating Jade).
\item Entwicklung von Datenbanken unter Einsatz von NoSQL-Technologie (MongoDB).
\item Konstrukteur der Infrastruktur, verantwortlich f{\"u}r die Bereitstellung (AWS, Docker, Heroku), Sicherheit (OAuth 2.0) und die Stabilit{\"a}t der Plattform.
\item Anforderungsingenieur. Ausgehend von den Anforderungen des Geschäftsbetriebs funktionelle Anforderungen erkennen.
\item Scrum Master. Planung von Sprints und verantwortlich f{\"u}r die Liefertermine.
\end{rSubsection}

%------------------------------------------------

\begin{rSubsection}{IBM}{August 2012 - Dezember 2012}{Technischer Analyst}{Bogot\'a D.C}
\item Ingenieur der Anwendungsintegration, beauftragt mit der Aufnahme der Anforderungen für die Integration und Implementierung der seibigen für ein Unternehmen aus dem Bankensektor. Die Arbeit entsprach einer traditionellen Entwicklungsmethode, die innerhalb des Reifegradmodells CMMI mit der Stufe 5 zertifiziert ist.
\item Implementierung von Integrationsdiensten unter Verwendung der Plattform Websphere Message Broker.
\item Implementierung von Java Adaptern für Legado Anwendungen.
\end{rSubsection}

%------------------------------------------------

\begin{rSubsection}{SoftOne SAS}{Juni 2012 - August 2012}{Freelance Entwickler}{Bogot\'a D.C}
\item Entwicklung von Webservices, aufbauend auf dem Framework JEE, für den Betrieb der Kanäle IVR und Kioskos eines Finanzunternehmens.
\item Entwicklung einer Anwendung f{\"u}r Mobiltelefone, aufbauend auf der Plattform J2ME, für die Verteilung von Auftr{\"a}gen.
\item Die zwei Projekte entsprachen einer flexiblen Entwicklungsmethode (Scrum).
\end{rSubsection}

[Anm. des Übers.: Ende der zweiten Seite des Lebenslaufs]

\end{rSection}


%----------------------------------------------------------------------------------------
% PERSONAL INITIATIVES
%----------------------------------------------------------------------------------------
\begin{rSection}{Pers{\"o}nliche Initiativen}

\begin{rSubsection}{Bogot\'a Lambda}{Juni 2016 - heute}{Mitorganisator}{Bogot\'a D.C}
\item Mitorganisator und Promoter eines Meetup im Zusammenhang mit Konzepten der funktionalen Programmierung.
\item Referent bei zahlreichen Meetups {\"u}ber Konzepte funktionaler Programmierung unter Einsatz von Scala und Haskell.
\item Entwickler von {\"U}bungsaufgaben, um Konzepte funktionaler Programmierung unter Einsatz von Scala und Haskell zu trainieren.
\end{rSubsection}

\end{rSection}

%----------------------------------------------------------------------------------------
% TECHNICAL STRENGTHS SECTION
%----------------------------------------------------------------------------------------

\begin{rSection}{Technische F{\"a}higkeiten}

\begin{tabular}{ @{} >{\bfseries}l @{\hspace{6ex}} l }
Softwarearchitektur & SOA (Services Rest und SOAP), DDD, MDD, Design- Muster \\
Programmiersprachen & Java, Scala, Pyhton, Javascript, Haskell \\
Bibliotheken & Scalaz \\ 
Frameworks und Plattformen & Akka, Spray.io, Play Framework, Spring, JEE, NodeJS, ExpressJS \\
Frontend & AngularJS, BackboneJS, CSS3, HTML5, Jade, Stylus, Grunt, Gulp \\
Protokolle \& APls & XML, JSON, SOAP, REST \\
Datenbanken & MySQL, PostgreSQL, MongoDB, Redis, Neo4j \\
Tools & Maven, SBT, Npm, Git, SVN
\end{tabular}

\end{rSection}

%----------------------------------------------------------------------------------------
% PROJECT MANAGEMENT SKILLS
%----------------------------------------------------------------------------------------
\begin{rSection}{Projektmanagementf{\"a}higkeiten}

\begin{tabular}{ @{} >{\bfseries}l @{\hspace{6ex}} l }
Methoden & Scrum, XP, TSP. \\
Kommunikation & Gute m{\"u}ndliche und schriftliche Ausdrucksf{\"a}higkeit.
\end{tabular}

\end{rSection}


%----------------------------------------------------------------------------------------
% LANGUAGES
%----------------------------------------------------------------------------------------
\begin{rSection}{Sprachen}

\begin{tabular}{ @{} >{\bfseries}l @{\hspace{6ex}} l }
Englisch & Gutes Sprech-, Schreib- und Leseverm{\"o}gen. \\
Spanisch & Muttersprache.
\end{tabular}

\end{rSection}

%----------------------------------------------------------------------------------------
% RECONOCIMIENTOS
%----------------------------------------------------------------------------------------
\begin{rSection}{Auszeichnungen}

{\bf Stipendium Ich m{\"o}chte studieren Uniandes} \hfill {\em 2009} \\
Universit{\"a}t der Anden, Boqota D.C. 
Verliehen an die hervorragendsten ICFES-Pr{\"u}fungsteilnehmer des Landes.

{\bf Bester Abiturient COLSEM} \hfill {\em 2008} \\
Schule Seminar San Juan Ap\'ostol.

{\bf Referent Javascript Konferenz, Medellin, Kolumbien} \hfill {\em Oktober 2013} \\
Zur Javascript-Jahreskonferenz eingeladener Referent, Aentr6pico S.A.S. vertretend. 
Thema: Explorer f{\"u}r Darstellungen, ein Tanz zwischen NodeJS und D3.js.

\end{rSection}


%----------------------------------------------------------------------------------------
% CURSOS
%----------------------------------------------------------------------------------------
\begin{rSection}{Kurse - Konferencen}


{\bf Coursera, Standford University, Vereinigte Staaten} \hfill {\em April 2016} \\
Maschinelles Lernen. Dieser Kurs bietet eine umfangreiche Einf{\"u}hrung in den Bereich des maschinellen Lernens, Datamining und in den Bereich der Erkennung statistischer Muster. Unter den zu behandelnden Themen finden sich: (i) {\"U}berwachtes Lernen (parametrischer/nicht-parametrischer Algorithmus, Support Vector Machines, Betriebssystemkerne, neuronale Netzwerke). (ii) Un{\"u}berwachtes Lernen (Clustering, Dimensionsreduktion, Recommender-Systeme, tiefes Lernen). (iii) Best Practices maschinellen Lernens (Bias-lVarianz-Theorie; Innovationsprozesse beim maschinellen Lernen und bei k{\"u}nstlicher Intelligenz).

{\bf Coursera, \'Ecole Polytechnique F\'ed\'erale de Lausanne, Schweiz} \hfill {\em Mai 2015} \\
Grunds{\"a}tze der reaktiven Programmierung. Kernelemente, um reaktive kombinationsfähiger Weise zu schreiben, und die Art und Weise, diese nachrichtengesteuerten Systemen anzuwenden, die skalierbar und resilient sind.

{\bf edX, Universidad de Berkeley, Kalifornien} \hfill {\em Septiembre 2010} \\
Artifitial Intelligence.
K{\"u}nstliche Intelligenz. Grunds{\"a}tze k{\"u}nstlicher Intelligenz.

\end{rSection}


\end{document}